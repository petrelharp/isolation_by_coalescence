%%%%%%
%%
%%  Don't reorder the reviewer points; that'll mess up the automatic referencing!
%%
%%%%%

\begin{minipage}[b]{2.5in}
  Resubmission Cover Letter \\
  {\it Molecular Ecology Resources}
\end{minipage}
\hfill
\begin{minipage}[b]{2.5in}
    Erik Lundgren \\
    \emph{and} Peter Ralph \\
  \today
\end{minipage}
 
\vskip 2em
 
\noindent
{\bf To the Editor(s) -- }
 
\vskip 1em

We are writing to submit a revised version of our manuscript, 
``Are populations like a circuit? 
Comparing isolation by resistance to a new coalescent-based method.''
(formerly titled ``Are populations like a circuit? The relationship between isolation by distance and isolation by resistance'').


...

\vspace{5em}

\noindent \hspace{4em}
\begin{minipage}{3in}
\noindent
{\bf Sincerely,}

\vskip 2em

{\bf 
Erik Lundgren and
Peter Ralph
}\\
\end{minipage}

\vskip 4em

\pagebreak
\setcounter{page}{1}

%%%%%%%%%%%%%%
\reviewersection{AE}

\begin{quote}
We received five reviews for your submitted article. While three are quite positive, two others are more critical. In the balance of comments, all reviewers felt that your article should be published in MER, though significant improvements were requested. Therefore, my decision is to request a revision before accepting your article.

Please respond to all major comments raised by the reviewers. In particular, try to address the following points: 1) Revise figures for clarity and legibility. 2) Include an empirical dataset, as suggested by REV 2. 3) Rewrite for a broader audience of biologists (translating some difficult concepts when possible and as appropriate). 4) Clarify aspects of the statistical methods as indicated by several reviewers.
\end{quote}

%%%%%%%%%%%%%%
\reviewersection{1}

\begin{quote}
    The authors present a new Bayesian method to fit movement rates between a
    set of demes to pairwise genetic differentiation, utilizing a MCMC. They first
    state the basic equations, which closely resemble existing theory of stepping
    stone models. They extensively compare their computational method to resistance
    based methods using an own implementation of the latter, both "under the model"
    as well as using various quite realistic forward in time simulations.
    Resistance based methods are much faster and can handle much more demes than
    the method presented here (no case with more than a 5x5 grid is shown).
    Therefore, as the authors point out, this work will likely not contest the use
    of resistance distances for visualizing spatial genetic patterns (which is
    often done in an exploratory way). But importantly, the method here can infer
    asymmetric and biased rates, unlike almost all existing methods, which presents
    a major advance -- and certainly fills a niche.  
\end{quote}

\begin{quote}
    Despite the computational limitations, I believe that this paper will provide
    researchers a good starting point for future development, by both giving a
    review about difficulties and also a first idea and example for an
    implementation.

    Overall, I think this article makes a very valid contribution for Molecular
    Ecology Ressources. The authors put their work into the right context and
    show good knowledge of the relevant literature. The writing is both clear
    and concise. Importantly, they do not shy away from pointing out the limits
    of their method, and the very stringent (often unrealistic) assumptions
    such methods usually make.

    I have several comments, but none of them are ``make-or-break'' or fatal flaws.
    The manuscript is already in a good state. But I hope the authors can use my
    comments to further improve their work.  
\end{quote}


\begin{point}{}
    Clearer point out the current computational limitations of your coalescence
time method. It seems it gets tricky with more than about 30 demes? What is the
run-time then, does the whole thing blow up? Give an idea how big is too big!
\end{point}

\reply{
    The runs with 16 demes take around 10 hours with the number of iterations we used
    (noted in the text, \revref).
    Thirty demes is probably a good limit for the current implementation. 
    There are changes that could be made to make it more efficient for larger systems, 
    but at that point the ill conditioning problem would be more severe 
    because there would be more parameters.
    We've added a note about this to the discussion. \llname{ll:computation}
}

\begin{point}{}
    What is the exact scaling? The bottleneck seems to be calculating the
    coalescence times with equation (7). You only say that it ``can be solved
    with standard linear algebra''. Are you doing Gaussian elimination or
    something else worse than $N^2$? Do you use any speed-ups to
    deal with the sparse nature of G for neighbor-only migration? Is it faster
    than getting the coalescence times by iterating time-discretized
    recursions?  It would be helpful for further progress to describe the exact
    scaling and the exact algorithm you use.
\end{point}

\reply{
    In the current implementation, 
    equation (7) is solved using \texttt{solve(A,b)} in R. 
    The actual method varies depending on the specific linear algebra library used, 
    but it is typically Gaussian elimination or LU decomposition, 
    both of which are $\mathcal{O}(N^3)$ 
    (unfortunately the decomposition is different in each step;
    if it were the same it would be $\mathcal{O}(N^2)$).
    For fine discretizations of space, the matrices are sparse,
    but in our testing, using sparse matrix methods didn't help at this scale.
    We've added a point about this last thing to the discussion \revref
    but haven't put in more detail (e.g., plots of scaling with $N$)
    because it depends on the details (sparsity of the adjacency matrix),
    and as we discuss above, we think that nonidentifiability is a more serious barrier
    to finer resolutions.
}

\begin{point}{}
    The title of the manuscript is a bit misleading. One would not guess that
    the introduction of a method to infer biased and asymmetric gene flow is the
    main point.  ``Isolation by distance'' does not appear often in the manuscript,
    neither its classic literature (Wright, Maryuama, Malecot, Barton etc.) nor
    direct investigation of genetic differentiation with geographic distance. But
    these things are also not the scope of this paper.  
\end{point}

% Citation:
% The Effects of Genetic and Geographic Structure on Neutral Variation
% Brian Charlesworth, Deborah Charlesworth and Nicholas H. Barton
% Annual Review of Ecology, Evolution, and Systematics
% Vol. 34 (2003), pp. 99-125

\reply{
    We do think a big point of the manuscript is discussion of the drawbacks of resistance-based methods,
    but we agree with the point about isolation by distance.
    We've retitled it: 
    ``Are populations like a circuit? 
    Comparing isolation by resistance to a new coalescent-based method.''
}

\begin{point}{}
    You could write an outlook stating that your method could be used to fit
    parameterized models with few parameters. For instance, in case the putative
    location of a barrier or an area of biased gene flow is known, one could try to
    infer it without fitting ALL movement parameters, but a simple parametric
    model, which would implicitly take care of the regularization.  
\end{point}

\reply{
    This is a good thing to emphasize. We've added a bit to this effect. \revref
}

\begin{point}{}
    You use mean absolute error in movement rate to describe the model fit. I
    believe this measure can be quite misleading for exponential movement rates.
    Large rates can be notoriously hard to infer (are you super or super-super well
    connected is almost the same), so it is easy to be off quite a bit, which
    drives the average absolute error in rate. Also in practice, one does not care
    so much about absolute rates but ``relative'' ones. So using a correlation
    measure R**2  of inferred movement rates with actual movement rates would be a
    very useful and natural fit parameter.  
\end{point}

\reply{
    This is a good point, and we did experiment with other measures of model fit.
    However, in practice the mean absolute error captured patterns as well as other methods.
    In the interest of not increasing complexity of the paper, we haven't included results
    from the other methods.
}

\begin{point}{}
    You write you use standard methods to check convergence and mixing of the
    MCMC chain \revreffull{1}{6}. Given the degeneracy and the high dimensionality,
    where there any problems? E.g., local optima you were stuck sometimes? Or
    was it always pretty straightforward? Maybe write a few sentences on that!
\end{point}

\reply{
    Convergence could be slow, and the likelihood surface is often ridgy, 
    but we did not see too many problems with getting stuck in completely wrong
    areas of state space.
    This the reason we introduced the non-standard ``pre-burn-in'' phase.
}

\begin{point}{\revref}
    $F_ST \neq$ Absolute divergence!!!  $F_ST$ is more a
    co-variance measure, and for instance heavily influence by sampling a batch
    of related individuals (whereas the mean of absolute divergence will be
    unchanged).
\end{point}

\reply{
    We can't agree more!!!!
    We've added a note about this, although not a lot of detail,
    because it seems outside the scope of this paper. \revref
    And, we thought we should mention $F_{ST}$ because it is commonly used
    in these situations.
}

\begin{point}{}
    You have your own implementation of resistance distance methods, by first
    extracting hitting times. However, you allow non-symmetric rates, something
    no existing resistance method would do. Given your observation that there
    is a lot degeneracy for resistance times (adding skew stuff to hitting
    times), this introduces uncertainty that likely cannot be broken such as
    with coalescence methods. This could drive the high uncertainty in Fig.\ 5
    upper right. Given your good results for Fig.\ 4 for the symmetric case,
    this is a bit surprising. Maybe run one scheme where you FIX to a single
    symmetric movement rate between each pair of demes. This setting would be a
    fairer comparison (and be much closer to what EEMS etc.\ actually do -- they
    do not even try to infer asymmetric rates).
\end{point}

\reply{
    We considered doing this, but
    decided to allow asymmetric migration for resistance distance
    in order to make the two methods more directly comparable.
    For instance, how would we compare to the truth if the truth is asymmetric?
    Comparing methods with both restricted to be symmetric 
    certainly could be useful and informative,
    but we chose not to include it because it seems not to be the most important comparison
    (as reality is probably often asymmetric)
    and we didn't want the paper to get overly complex.
}


\begin{point}{}
    Maybe describe the model you use for you calculation and inference, a
    continuous backward in time Markov model with coalescence rates in demes
    before the actual recursions, and not bit by bit in there.
\end{point}

\reply{
    We've added a bit more to our big-picture overview that we hope achieves
    the goal of setting the stage appropriately. \revref
}

\begin{point}{\revref}
    Do you actually use this tensor product for the movement
    rates anywhere? And seems Eq.\ 7 (which has no tensor product) is all you
    need.
\end{point}

\reply{
    We included this because we think it's a nice compact way to write, code, 
    and think about the equation,
    and the tensor product equation is actually the one used in the R code.
    It is correct that the equations are equivalent.
}

\begin{point}{Line 115:}
    Not all $ZC$ skew-symmetric would translate to valid generators 
    $G + Z$. The diagonal elements of G have another condition (they have to be minus
    the sum of all other jumping rates). Would be interesting to explore this a
    bit more -- overall what the dimensionality this degeneracy is (implicit
    function theorem), and also if there are simple toy examples where they pop
    up.
\end{point}

\reply{
    Thank you for catching this.
    You are correct that in addition to $ZC$ being skew-symmetric,
    the rows of $Z$ must also sum to zero, which we now say. {\revref}
    Further exploration of the constraints would be interesting,
    but we think is beyond the scope of this paper.
}

\begin{point}{Eq. 4:}
    $H$ may not be always invertible ($H$ symmetric and positive off-diagonal
    and 0 on the diagonal is NOT enough). So there might be some pathological
    cases where multiple G link to one $H$, maybe even in some simple toy
    examples.
\end{point}

\reply{
    Good point; we had not justified this.
    Under the standing assumption that the chain is irreducible, 
    $H$ is always invertible.
    We now prove this in the Appendix \llname{rr:H_invertible},
    and have made reference to this fact. \revref
}

\begin{point}{}
    Dimensionality arguments (Line 129 onward) can be a bit misleading when you
    already know the equations have a non-trivial kernel. Setting all
    non-neighboring elements of $G$ to 0 might not get rid of parts of the
    kernel. So maybe flag it even more as ``be cautious'' as you already have.
\end{point}

\reply{
    Good point; we've changed the wording here. \revref
}

\begin{point}{}
    You could just call the thing by its name: ``Stepping stone'' model. This
    term has such a rich history in population genetics!
\end{point}

\reply{
    In our experience, the term ``stepping stone model'' is used in a variety of ways in the literature
    (e.g., also for all-migrate-to-all island models),
    so we prefer to describe explicitly the grid that we use.
}

\begin{point}{}
    In the main text, you never mention that the mean of the exponential
    movement rates for simulations is one, only in a figure. One can also kind
    of see it from the model figures, but it would be nice to have it
    explicitly stated in the main text.
\end{point}

\reply{
    Good point; we've added this. \revref
}

\begin{point}{}
    In the figures, probably keep model figure and inference results closely
    together. This helps the reader, and makes the figures stand more for
    themselves.
\end{point}

\reply{
    This is a good idea,
    but we haven't been able to see how to implement in a way that would improve things.
    Maybe we don't see your idea,
    but the additional graphs that describe population structure in the Appendix
    won't be legible if added to existing figures,
    and we think we already have more than enough figures.
}

\begin{point}{}
    Useful result to know: Maryuama’s invariance principle (Nagylaki 1998):
    That the correctly weighted within coalescence time within demes only
    depends on the total population size  (given conservative migration) -- so
    it cannot fluctuate too wildly across demes and movement rates! This fact
    motivates why not fitting too wildly differing within-deme coalescence
    times/diversities.
    % Nagylaki, T., 1998 The expected number of heterozygous sites in a subdivided
    % population. Genetics 149: 1599
\end{point}

\reply{
    Good reference! We'd already referred to the simpler version of \citet{strobeck1987average},
    and have now added this reference. \revref
}

\begin{point}{\revref}
    400,000 generations will look a bit lunatic for someone not in
    popgen who does not know about rescaling. You could give re-scaled, more
    realistic values!
\end{point}

\reply{
    We didn't really intend that number to sound so outlandish,
    but we think it's more ``quite long'' than ``absurdly long''?
    We're not sure that mentioning rescaling will help,
    but have added a reassuring note. {\revref} 
    Note that we \emph{did} actually simulate for that many generations;
    we didn't use rescaling to run a shorter or smaller simulation.
}

\begin{point}{Paragraph beginning \revref:}
    Maybe move this section in front of the inference part, as it is useful to
    know the most ``fundamental'' limit of resistance distances.
\end{point}

\reply{
    This is a good suggestion, 
    but we'd rather keep the different results having to do with the same model
    together in the same place.
}


\begin{point}{Fig. 7:}
    With EEMS one can see very suspiciously the 4x4 sampling structure - there
    are red dots in almost all 3x3 centers of the sampling squares. Since EEMS
    is quite fast, could you increase the resolution of sampling?
\end{point}

\reply{
    More samples could be added for the EEMS runs, 
    but doing so would have the drawback 
    that the two methods would no longer be using the same data for inference. 
    We've been told that similar patterns show up when EEMS tries to approximate
    anisotropic migration,
    so it may not be the sampling pattern.
    We've added a note about this. {\revref}
    Also, note that these samples are more spread out that people often have in practice:
    they do occur in the middle three-quarters of each grid square,
    but they aren't all located at the center.
}

\begin{point}{}
    You could experiment with visualizations a bit. EEMS got quite popular as
    exploratory visualization tool because it produces visually pleasing plots.
    Do lines with two arrowheads of different sizes near the middle, maybe even
    colored (blue to red) work?
\end{point}

\reply{
    We agree -- there's a lot of directions we could move in to improve our package,
    and hope to work more on this.
}



\reviewersection{2}


\begin{quote}
    This work revisits McRae’s seminal isolation-by-resistance (IBR) model through
    the lens of coalescent theory. The original idea of IBR was to compare patterns of
    genetic distance observed on a landscape with the so-called ``resistance distance''.
    The resistance distance between states $i$ and $j$ in a Markov chain is defined to be
    the ``commute time'': the expected amount of time it takes for the chain to go
    from $i$ to $j$ and back, integrated over all possible paths of the chain. Resistance
    distance comes from electrical circuit theory and is not a model of genetic
    variation across a landscape. Also, it is inherently symmetric in $i$ and $j$, and
    cannot accomodate asymmetric migration between demes. A more biologically
    realistic model is the structured coalescent. Hence, if data are generated under
    the something resembling the coalescent, then there is scope for improving IBR
    by reducing model misspecification.

    This paper implements such a method and conducts comparative analyses on a
    number of simulated data sets. It seems that in general, this method and the
    resistance-based approaches (as exemplified by EEMS) produce somewhat com-
    parable results so long as migration rates are symmetric. The largest differences
    emerge when gene flow is asymmetric, because this fundamentally violates the
    resistance distance model used by EEMS. Here, the authors show convincingly
    that inferring migration rates using the resistance distance approximation can
    be wildly inaccurate compared to the same estimates obtained using coalescent
    methods. To the extent that gene flow is more likely to be asymmetric in many
    natural populations, this seems like an important finding.

    My overall impression of this work is fairly positive. The paper is very clearly
    written, and the idea of replacing IBR with a method that is more faithful to
    the underlying biology is appealing. The methods section (as well as appendices)
    feature a very nice and readable summary of the mathematical innovations of
    this work, as well as some interesting discussion of issues of identifiability and
    power. Similarly, discussion section in very well done and really strengthens
    the manuscript. It includes a thorough survey of the related work, as well as a
    forthright discussion of advantages and disadvantages of the proposed method,
    and it raises a number of interesting questions and directions for future research.
    Overall, the paper was a pleasure to read.
\end{quote}

Thanks very much for the kind words.
One minor comment: resistance distances are not strictly based in a population genetics model,
but we think that in practice it is thought of as (an approximation to) such a model
-- for instance, it is common to see the explanation that it
``integrates over all possible paths that the organisms might take''.
We think it is clear that \citet{mcrae2006isolation} saw resistance distance as an approximation
to the more-accurate coalescence models; we're looking at the effects of this approximation.
We hope this point comes through in the paper.


\begin{quote}
    I find the main deficiency to be the lack of any analysis of real data. All
    experiments are carried out on simulated data only. This is a puzzling omission.
    The authors went to the trouble of creating an R package (which is good), so
    clearly they intend for the method to be used in practice. Personally, I would
    hesitate to publish the results obtained from a method which had not previously
    been verified to produce reasonable answers in real data. This really feels like
    something the authors ought to have included. It is not necessary that the data
    be novel -- reanalyzing (say) the African elephant dataset in the EEMS paper
    would provide for an interesting comparison (even if it cannot be done at an
    equivalent spatial resolution) and would also go long way towards reassuring
    practitioners.
\end{quote}

We've now added analysis of a real dataset, from \citet{geraldes2014landscape},
and are quite pleased with the results!
This has also given us a chance to spend more time talking through how to interpret the results.
(We didn't get to say this in the paper,
but initially were worried by the southward bias in gene flow for the inland poplar species,
since we would expect the opposite if poplars had expanded to the north following the glaciers;
but then we learned that this species likely expanded from Alaksa, rather than from the south.)


\begin{point}{}
    ``Bayesian inference of movement rates'': In their model the authors assume
    that many entries of the migration rate matrix are zero, and place an
    exponential prior on the rest. Is it safe to assume that we will always
    want to constrain rates between discontiguous cells to zero? Also, given
    the valid concerns that they raise elsewhere about identifiability and
    power, I wonder if the authors considered placing a sparse prior on the
    entire G matrix, enabling the model to adapt to the amount of signal while
    shrinking other migration rates to zero. This might have the effect of
    making the resulting non-zero estimates more stable. (I realize that this
    raises potential problems for inference if the underlying graph becomes
    disconnected.)
\end{point}

\reply{
    Whether or not it is reasonable to assume there is no long distance migration
    depends on the population under study. 
    Putting a sparse prior on all of $G$ is a good idea for future improvement.
    % although it would probably be appropriate for the prior probability of a value
    % to be non-zero to be much higher for adjacent demes as compared to non-adjacent ones, 
    % which would also likely prevent the graph from becoming disconnected very often.
    (Since disconnecting the graph would result in some infinite coalescence times, 
    such proposals would be rejected by the MCMC algorithm.)
}

\begin{point}{}
    The discussion of ill-conditioning is welcome. From experience, it often
    feels like coalescent-based methods have little power to accurately
    estimate migration unless the rates are large. It might be interesting to
    shift emphasis from estimation to the slightly easier problem of testing:
    how good are these methods at distinguishing between a given null and an
    alternative hypothesis (say, barrier vs.\ no barrier) for the rate matrix $G$?
    I think the gain of using a coalescent model might be more apparent in this case.
\end{point}

\reply{
    This is a good point; we've added it to the discussion
    (and credited you with it). \revref
}

\begin{point}{}
    This method is computationally limited by equation (7), which has a
    complexity of $O(d^6)$ for the product Markov chain on d demes mentioned a
    few lines below. At least, that is the cost for solving (7) exactly -- but
    given the noisiness of the matrix C, there may be little point in doing
    that anyways. I wonder if the authors have thought at all about obtaining
    an approximate solution to (7), where the error in approximation is chosen
    to in some way to balance out the statistical error inherent in C / D. It
    may also be possible to leverage the fact that the solution G is sparse
    here.  (This is more of a suggestion for future research rather than a
    suggestion for revision.)
\end{point}

\reply{
    Thanks for the suggestions.
    Solving (7) directly for $G$ and $\gamma$ with a noisy $C$ is extremely difficult
    even in cases small enough to just use least squares; 
    it typically resulted in some negative values unless there is very little noise in $C$.
    In order to solve for $C$ from $G$ and $\gamma$, 
    we experimented with the Jacobi method. 
    While it seemed to converge fairly reliably, 
    at the grid sizes where it was computationally advantageous to use it over solving exactly, 
    it would still take a very long time for the MCMC to converge
    due to the large number of parameters.
    % Devoloping a way to reduce the number of parameters in a way similar to EEMS
    % while still allowing asymmetric migration is a good direction for future implementations.
}

\begin{point}{\revref}
     Do the authors have any other guidance on how to discretize the landscape?
\end{point}

\reply{
    This is a large, thorny issue,
    closely related to the question of how to translate discrete-population models
    to continuous geography.
    We've added a bit more \revref,
    as well as some more discussion of this in the section on poplar data.
    \llname{ll:discretization}
}

\begin{point}{}
    Does the issue of having negative entries of G (lines 112-117, Appendix A)
    arise during the course of MCMC sampling, and if so, how is it dealt with?
\end{point}

\reply{
    No. The proposal distribution includes taking the absolute value of parameters
    (a standard method in MCMC to preserve reversibility).
    We've added this note to the MCMC methods section. \revref
}

\begin{point}{\revref}
     suggest ``commonly'' in place of ``not uncommonly''
\end{point}

\reply{
    Changed.
}

\begin{point}{\revref}
     strongly misleading?
\end{point}

\reply{
    We think this is correct (and reads reasonably well?) as it stands.
}


\reviewersection{3}

\begin{quote}
    This is an interesting paper on a topic that should be of wide interest to
    readers of MER.  The main results are useful: 
    \begin{enumerate}
        \item      landscape genetics models based on resistance theory can be
            misleading if gene flow is asymmetrical
        \item      coalescent-based models deal better with asymmetry but are
            not a panacea for a variety of reasons.
    \end{enumerate}
    On the other hand, I found myself with quite a few questions after reading
    the manuscript.  Raising new questions is not bad in and of itself, but
    most of these were questions that could/should have been clarified in the
    text.  As a consequence, the manuscript is not as effective or useful as it
    might be.  The problem is not in the writing itself, which generally is
    careful and precise.  Instead, the main problem involves translating a very
    technical exercise into something that is accessible to typical readers of
    the journal.  Some examples are described below.

    Although the theme of the paper is relevant to the journal, I think most
    readers of MER would find the manuscript hard going in many places.  For
    example, the text at 102-109 and 136-146 would benefit from translation
    into insights more meaningful to biologists.  The authors do a good job of
    this in discussing results for asymmetric gene flow at lines 296-299. % \llname{ll:dense3}
\end{quote}

Thanks for the encouragement.
We've tried to improve accessibility in a number of places,
in particular by adding a ``flag'' at the top of the methods {\llname{ll:flagging}},
and improving the two bits of text referred to,
at {\llname{ll:dense1}} and {\llname{ll:dense2}}.


\begin{quote}
    The reader also needs to understand the biological consequences of various
    assumptions.  Examples  where this could be improved include:
\end{quote}

\begin{point}{\revref}
      if the genome is selectively neutral ...
\end{point}

\reply{
    Reworded.
}

\begin{point}{\revref}
     as long as long distance migration is disallowed
\end{point}

\reply{
    Added parenthetical clarification.
}

\begin{point}{\revref}
     $C$ is large compared to $\sigma$
\end{point}

\reply{
    Reworded.
}

\begin{point}{\revref}
     we can estimate amount of noise well
\end{point}

\reply{
    Added explanatory note aobut how to estimate noise.
}

\begin{point}{}
    The text at 468ff relating to the assumption of constant pop size and migration
    rates does a good job of explaining these consequences.  Perhaps all
    assumptions could be listed in a single place and their biological
    consequences explained.
\end{point}

\reply{
    The job of enumerating \emph{all} assumptions involved in modeling a real population
    with a system of discrete demes is daunting.
    We've put in more discussion of these pitfalls (as described above),
    but a comprehensive review is beyond the scope of this paper.
}

\begin{point}{\revref}
    How exactly is the pairwise genetic distance calculated?  What does ``density
    of sites'' mean?  Is it just the fraction of sites at which two genomes
    differ?  If so it is just pairwise theta as used by Wakeley and others.  
    Eq 1:  what is $R_{ij}$ and how is $q$ calculated?
\end{point}

\reply{
    That's correct; we've reworded this to be more clear.
}

\begin{point}{}
    Eq \eqref{eq:commute_approx}: what is $R_{ij}$ and how is $q$ calculated?
\end{point}

\reply{
    We've added a clarifying sentence. \revref
}

\begin{point}{}
    $G$, the movement rate, is clearly a key parameter but is not clearly defined (or
    if it was I missed it).  What are the units of G?  Geographic distance per
    unit time?  What are the time units?  Years?  Generations?  If the latter,
    then does G represent parent-offspring dispersal distance as defined by
    Wright?  Are generations assumed to be discrete?  If not, can a migrant
    leave offspring at various different geographic points?
\end{point}

\reply{
    A precise definition (in words and equations) is given where it is first introduced, at \revref.
    We've rewritten it to be hopefully more clear.
}

\begin{point}{}
    As described, the model deals with groups of individuals (populations) that
    sometimes exchange individuals.  But much of landscape genetics/genomics
    deals with continuously-distributed species that don’t have recognizable
    populations.  Are the results also applicable to scenarios like this?  Why
    or why not?
\end{point}

\reply{
    This is an important point that applies not just to our method
    but to most methods of population genetics inference,
    and we view this disconnect as a major unsolved problem in the field.
    Our simulations of continuous space are explicitly designed
    to look at this problem.
    We've added substantially more discussion of this point
    in the interpretation of poplar results \llname{ll:continuous}.
}

\begin{point}{}
    The authors identify two sources of noise: Sampling noise (arising from
    sampling of individuals) and Process noise (stochasticity of population
    history).  Using SLiM, which I gather controls for random variation in the
    realized pedigree and hence eliminates (or only reduces?) Process noise,
    the authors obtain estimates for the relative magnitude of the two sources
    of noise.  This is useful but seems incomplete for two reasons:
    \begin{enumerate}
        \item       A third source of noise is sampling a finite number of
            genes or SNPs.  To some extent one can trade off sampling
            individuals vs loci, but there are limits to how effective that is.
        \item       The relative magnitudes of the three sources of error are
            not constant and can vary dramatically depending on parameter
            values.  Process error is inversely related to $N_e$ but otherwise
            fixed in magnitude; it does not change when sampling errors change.
            This means that Process error becomes relatively more important
            when there is a lot of replication in sampling.  In particular,
            genetic sampling error gets very small as the number of SNPs
            becomes arbitrarily large, in which case it is dwarfed by Process
            error (e.g., see Figure 6 in our 2009 Mol Ecol paper cited by the
            authors).
    \end{enumerate}
\end{point}

\reply{
    We certainly had omitted that source of noise,
    but have added it to the list now \revref.
    We agree it can be substantial, but aren't focusing on it here,
    as the contribution of sampling noise is much less appreciated,
    particularly in the context of continuous space (see our additional note on this).
    We're not sure what you mean about SLiM reducing process noise:
    on the contrary, since we use SLiM to simulate in continuous space,
    we expect process noise to be larger than in a randomly mating population.
}

\begin{point}{}
    Also in relation to genomics-scale datasets, two general potential
    advantages can be identified:
    \begin{enumerate}
        \item       Arbitrarily large numbers of loci can increase precision
            and statistical power for existing analyses;
        \item       Ability to access information related to genome structure
            and function can provide novel insights and make possible
            qualitatively new types of analyses
    \end{enumerate}
    Lots of people have genomics-scale datasets these days so some discussion of
    this issue seems important.
\end{point}

\reply{
    Good point, but to keep the paper relatively focused we're not going into this in detail
    (but perhaps the pointer at {\revref} suffices?).
}

\begin{point}{}
    ``Can our method replace resistance-based methods? Perhaps, but the substantial
    uncertainty we saw on graphs with only tens of nodes is indicative of a
    larger problem we face for realistic models. We have seen that
    discretization of space results in substantial modeling error, in part
    because of randomness of geographic sampling and unmodeled process noise
    can lead to overfitting. Partitioning space into a finer grid should help
    with these problems, but tends to make the inference problem itself more
    ill-conditioned: with more connections, changing the value of one
    connection affects coalescence time less, and so inferences about that
    value must necessarily be less certain. This tradeoff implies some degree
    of unavoidable uncertainty.  Reproducible, reliable inference will likely
    require development of new inference methods that explicitly model
    continuous geography.''  This seems like a fairly crucial paragraph.  But
    what exactly is the take-home message for biologists interested in
    practical application of the methodology?
\end{point}

\reply{
    Good question!
    As written, the take-home was ``wait until we make better methods'',
    but we've added a better one. \revref
}

\begin{point}{}
    Coalescent rates are critical to model performance, which can be poor if the
    rate is too low.  It is obvious that the coalescent rate is inversely
    related to $N_e$, but in practical terms, what does it mean when the
    coalescent rate is lowered from 10 to 0.1?
\end{point}

\reply{
    We've added more context to this section. \revref
    % Assuming that the movement parameters are of order one, 
    % a coalescence rate of 10 means that 
    % lineages would be more likely to coalesce the first time they are in the same location
    % and a coalescence rate of 0.1 means that lineages would likely meet many times
    % before finally coalescing. 
    % For example, if the gene flow rates out of a deme for both lineages sum to 8
    % (rate of 1 for each direction for each lineage), 
    % the chance of the lineages coalescing before one of them leaves 
    % would be 10/18 (~56\%) with a coalescence rate of 10 
    % but only 0.1/8.1 (~1.2\%) with a coalescence rate of 0.1.
}

\begin{point}{\revref}
    ``we varied the standard deviation of the noise from 1/1000th to 1/50th of
    the mean value of $C$ for that grid.''  This sounds like a CV of 0.001 to
    0.02 for noise, both of which would be exceedingly small.  If that is not
    what is meant, please clarify.
\end{point}

\reply{
    That is what we mean, and although small, we don't think that's unreasonable for genome-scale data
    under a discrete population model (but see our later discussion of process noise).
    We realize that many people are working with much smaller number of (ascertained) loci,
    but there's enough bigger issues here related to continuous geography
    that we've not explored in that direction.
}

\begin{point}{\revref}
    ``if fecundity is variable'' -- on what scale?  Among populations?  Among
    individuals?  Among ages within the same individual? 
\end{point}

\reply{
    We mean among individuals (and have added this).
}

\begin{point}{\revref}
    ``It is not required in this approach to have samples from every spatial
    location \ldots''  Required for what?  Are they saying that unsampled but real
    (ghost) populations have no effect on precision or bias?
\end{point}

\reply{
    We're simply saying that it is not a requirement to \emph{run} the method
    to have samples in each location --
    it is easy, in principle, to include ``ghost'' populations.
    Of course, unsampled populations will affect accuracy of the result,
    and we've added a note to clarify. \revref
    % In practice, the more locations with samples you have, the better the result will be.
    % We would not typically recommend doing inference that includes unsampled demes
    % if it could be avoided.
    % However, in the theoretical case of a population fitting the model perfectly
    % and knowing the other genetic distance values perfectly, 
    % it is possible to solve exactly for paramters relating to the unsampled populations
    % as long as there are as many unique equations as unknowns.
    % In a real situation, if data is very good for the sampled locations, 
    % inference should be reasonable for the unsampled ones.
}

\begin{point}{}
    Many of the figures are too small to read easily without a magnifying glass.
    Figures 3-5 are large enough but they include miniscule fonts that make it
    impossible to tell exactly what is plotted.  Green and black do not provide
    enough contrast.
\end{point}

\reply{
    We sympathize, and have tried to improve this.
    However, some of this we think is unavoidable --
    the x-axis labels of figures like Figure~\ref{fig:5x3b_post_coalvcom}
    cannot be made larger without overlapping.
    Another approach would be to simply plot true versus observed values,
    which is equivalent to the current plot without the x-axis labels.
    We prefer to leave these labels, even if these are only really legible
    in the digital version (after zooming in).
}


\begin{point}{\revref}
    another limitation of direct observations is that they only provide
    information about movement of individuals but not gene flow, which requires
    successful breeding
\end{point}

\reply{
    Good point, but we'd like to keep this bit more concise.
}

\begin{point}{\revref}
    the ``wrong answer'' compared to what?
\end{point}

\reply{
    The truth. We've changed the phrasing. \revref
}

\begin{point}{\revref}
    What is EEMS?
\end{point}

\reply{
    A lengthy name (that we've added). \revref
}

\begin{point}{\revref}
    What exactly is a more general problem?
\end{point}

\reply{
    Clarified. \revref
}

\begin{point}{\revref}
    need a semicolon, not a comma, after ``two genomes''
\end{point}

\reply{
    We reworded this.
}

\begin{point}{\revref}
    why quotes for ``from the model''?
\end{point}

\reply{
    Removed. \revref
}

\begin{point}{}
    ``suppose that the spatial arrangement of populations is given by a graph
    \ldots''  ``Graph'' is a very general term and not clear exactly what is meant
    here.  If the word has a more specific meaning in this context, please
    define it.
\end{point}

\reply{
    Done. \revref
}

\begin{point}{Fig 6 caption:}
    the barrier is ``sufficiently thick'' to prohibit migration.  ``Thick''
    implies distance, but is that a requirement?  Couldn’t the barrier be
    higher or deeper or less penetrable for a different reason than thickness?
\end{point}

\reply{
    Yes, but this is the way we implemented it in the simulation.
}



\reviewersection{4}

\begin{quote}
    This is a polished theoretical and methodological contribution towards
    understanding the genetic processes that affect genetic distance and thus
    connectivity of species across the landscape. The authors make a
    little-appreciated point about considering the coalescent time as a
    predictor of genetic distance as opposed to resistance distance which can
    result in spurious results  where genetic distance is shaped by
    asymmetrical gene flow and coalescent processes.

    There are some points of clarification however that I think need to be made
    so that the paper is accessible and clear to its intended audience, and to
    ensure the paper can be adopted and influence analyses in this area. The
    differences in the highly spatial resistance distance approach versus the
    genetics-based coalescent time-based approach need to be distinguished more
    clearly, as in some ways this may be seen as comparing apples and oranges.
    Issues around timescales of genetic change (effects of landscape vs
    coalescent times) need to be further addressed and elucidated.  The
    differences in the results gained from the approaches are clearly defined
    and presented however.

    The types of questions and the type of demographic scenarios under which
    resistance vs. coalescense based approaches may be an improvement, or
    combined in the same study should be made more clearly. I suggest a
    summarising table or flow chart to this effect, that addresses time scales
    also.
\end{quote}

We've added more discussion along these lines
particularly in {\llname{ll:history}} 
and in relation to analysis of the poplar data.
We've got quite a few figures already (and have added one for the poplars),
so haven't tried to add a flow chart.

\begin{point}{Line 7 of abstract and generally:}
    This statement is conditional on the characteristics of the data you have, and
    your study aim. If the study aim is to  ask how landscape features affect
    genetic distance, then the coalescent time is not going to explicitly test
    this, or? It needs to be made clear how different study objectives will
    benefit in landscape genetics and under what circumstances a `more
    accurate' predictor of genetic distance can be gained. It seems appropriate
    to mention that this approach would greatly complement existing landscape
    genetic analysis approaches to see the relative effects of resistance and
    coalescent time. Implications for management decisions based on resistance
    models could be mentioned, as resistance models are intended to inform
    conservation decisions in some cases.
\end{point}

\reply{
    We're not sure which possible study aims are being referred to,
    but one possible aim that is not possible currently with coalescence methods
    is to produce high-resolution maps (as we discuss, \revref).
    We made the statement that
    genetic data are more accurately modeled by coalescence time than resistance distance
    because while there is an extensive literature going back fourty years
    describing the broad conditions under which coalescence is a good model 
    for the process that produces genomic data,
    we know of no such justification for resistance distance.
    Perhaps the key disconnect here is that we're thinking about generative models?
    We agree that since coalescence theory does not always apply,
    resistance distance on some network could in principle be closer to the truth
    than any coalescence times.
    However, we're unaware of this claim having been made.
    Furthermore, as we understand it,
    the point of landscape genetics methods using resistance distance
    is not merely to recapitulate observed genetic distances,
    but to provide an accurate depiction of past or present gene flow,
    i.e., movements of individuals across the landscape,
    In our paper we've tried to describe how inaccuracies of the generative model
    that underlies resistance distance
    can lead to misleading results.
    As discussed at \llname{ll:approx},
    resistance distance is used only because it is a more computationally tractable
    approxmation to coalescence time --
    any method that uses resistance distances could in principle be 
    done with coalescence times instead (e.g., to look at the effect of landscape features).
    This is discussed at \llname{ll:circuitscape}.
}

\begin{point}{\revref}
    Extremely long sentence, please divide.
\end{point}

\reply{
    Done.
}

\begin{point}{\revref}
    Would it make sense to select genetic markers that are more informative for
    coalescence rates by narrowing the level of variation accepted in the
    model? Would this help error rates justifiably or just bias interpretation?
\end{point}

\reply{
    This is an interesting idea,
    and suggests looking at a statistic of genetic differentiation other than the mean,
    perhaps IBD segment lengths (as done by \citet{alasadi2018estimating}).
    We don't know enough about this to speculate, however.
}

\begin{point}{\revref}
    I think it  might be helpful if this point was made earlier and reiterated
\end{point}

\reply{
    Done, at \llname{ll:threepop}.
}


\reviewersection{5}

\begin{quote}
    This article focuses on the inference of gene flows between populations from
    the analysis of georeferenced genetic data. In particular, a new
    coalescent-based technique, which applies to the analysis of pairs of
    sampled individuals, is compared to an approach based on "resistance
    distances" which aim at approximating coalescent distances. This work is
    interesting because resistance distances are at the core of several popular
    techniques in landscape genetics. It is thus important to gain a good
    understanding of how good this phenomenological approximation is compared
    to more mechanistic techniques such as the structured coalescent. The
    article clarifies the connections between the two approaches in a
    convincing manner. It thereby fully deserves to be published in Molecular
    Ecology Resources. There are however a few points (see below) that I think
    should be improved should the authors see fit.
\end{quote}


% Section "Hitting times of Markov chains"

\begin{point}{}
    $Pr(X_{t+e}=y | X_t=y)$ should read $Pr(X_{t+e}=y | X_t=x)$.
\end{point}

\reply{
    Whoops, thanks! \revref
}

\begin{point}{}
    Regarding Equation 2, I would recommend giving more detail explaining where
    this comes from. Here is a suggestion:
    \begin{align*}
        H_{xz} &= dt + \sum_y Pr(X_{dt} = y | X_0 = x) H_{yz} \\
               &= dt + \sum_{y != x} Q_{xy} dt H_{yz} + (Q_{xx} dt + 1) H_{xz}
    \end{align*}
\end{point}

\reply{
    We wondered whether this would overload readers,
    but have now included an explanation along these lines.
    \revref
}

\begin{point}{}
    On a related note, it might be helpful to give more explanations about Equation 6. In
    fact, I am not quite sure I agree with it. Indeed, for $x \neq z$, we have
    \begin{align*}
        C_{xz} &= dt + {} \\
                & \qquad
             1/2 \sum_y Pr(X_{dt} = y | X_0 = x) C_{yz} + {} \\
                & \qquad
             1/2 \sum_y Pr(X_{dt} = y | X_0 = z) C_{xy}, \\
           &= dt + {} \\
                & \qquad
             1/2 \sum_{y \neq x} Q_{xy} dt C_{yz} + (Q_{xx} dt + 1) C_{xz} + \\
                & \qquad
             1/2 \sum_{y \neq x} Q_{zy} dt C_{xy} + (Q_{xx} dt + 1) C_{xz},
    \end{align*}
    where the 1/2 terms correspond to the probability of "selecting" the lineage found in x
    (resp. z) at time 0 to move to y after dt. The last equation above then gives
    \begin{align*}
        -2 = \sum_y Q_{xy} C_{yz} + Q_{zy} C_{xy}
    \end{align*}
    which differs from the one given in the ms. I have probably missed something in the derivation
    used by the authors. Giving a few more explanations here might help avoiding any confusion.
\end{point}

\reply{
    Besides substituting $G$ for $Q$,
    our equations only differ by the factor of 1/2.
    The model we work under is that each lineage moves as an independent copy
    of the Markov chain $X$, so that the total jump rate
    when the two chains are at $x$ and $z$ is $-G_{xx}-G_{zz}$.
    We think the factors of 1/2 are in error --
    a more clear way to see this would be to say that $X$ and $Z$ are the locations of the two lineages,
    that move independently until they coalesce,
    and so
    \begin{align*}
        C_{xz} &= dt + {} \\
                & \qquad
             \sum_u \P(X_{dt} = u | X_0 = x) C_{uz} + {} \\
                & \qquad
             \sum_v \P(Z_{dt} = v | Z_0 = z) C_{xv}, \\
           &= dt + {} \\
                & \qquad
             \sum_{u \neq x} G_{xu} dt C_{uz} + (G_{xx} dt + 1) C_{xz} + \\
                & \qquad
             \sum_{v \neq x} G_{zv} dt C_{xv} + (G_{xx} dt + 1) C_{xz} .
    \end{align*}
}


% Section "Bayesian inference of movement rates"

\begin{point}{}
    I was also a bit puzzled by the likelihood function that is at the core of the
    Bayesian inference proposed here (see Equation 8). The spatial component of
    the data seems completely hidden. It is in fact conveyed by the coalescent
    times $\mathcal{C}_{ij}(G,\gamma)$ but a more detailed description of the likelihood
    function where both genetic and spatial information appear explicitly would
    facilitate the understanding of the proposed inference technique.
\end{point}

\reply{
    We've added a clarifying sentence. \revref
}

\begin{point}{}
    Moreover, my understanding is that the joint probability of the pairwise genetic distances
    ($D$) and spatial locations ($L$) given $G$ and $\gamma$, i.e.,
    $p(D,L|G,\gamma)$, is expressed as the product 
    $p(D|L,G,\gamma) \times p(L|G,\gamma)$.  Thanks to Equation (7) the first density is in fact
    $p(D|C)$, where $C$ are the coalescent times. The authors use a normal distribution for $D|C$ but
    nothing is said  about the distribution of $L|G,\gamma$. It is paramount to explicitly define
    every probabilistic distribution in the likelihood of the model used for the inference so as
    to clarify what are the underlying simplifying hypotheses the proposed approach relies on.
\end{point}

\reply{
    In our model, spatial locations are (discrete and) assumed to be known without error,
    and we do not want to model the sampling process,
    so they do not enter into the likelihood.
    Also recall that both $D$ and $C$ are \emph{complete} matrices of genetic distances 
    and pairwise coalescence times -- i.e.,
    between all pairs of populations --
    so the locations do not enter at this stage (although they are certainly used in estimating $D$).
    We have added a note reminding the reader of this. \revref
}

\begin{point}{}
    On a related note, my understanding is that the uncertainty affecting
    coalescent times conditional on the movement rates is not taken into
    account in the proposed approach (i.e., knowing $G$, one gets $C$ through
    Equation 7 in a deterministic fashion). I would recommend that the authors
    discuss this point.
\end{point}

\reply{
    This is true; we've added a sentence emphasizing this. \revref
}


% Section "Identifying a barrier to gene flow"

\begin{point}{}
    The sentence ``with noise standard deviation equal to 1/1000th of the mean
    coalescence time (left) and 1/100th of the mean coalescence time (right)''
    should read ``with noise standard deviation equal to 1/1000th of the mean
    coalescence time (top) and 1/100th of the mean coalescence time (bottom)''
\end{point}

\reply{
    Whoops! Fixed. \revref
}


% Section "Discussion"

\begin{point}{}
    The authors call for new models to be developed that would properly accommodate
    for individuals distributed in space along a continuum as opposed to
    discrete demes, as is the case with the structured coalescent. The spatial
    Lambda-Fleming Viot model is one of these. It has been studied extensively
    from a mathematical standpoint (see the body of work by Etheridge, Barton
    and Veber amongst others) and has been used for the inference using a
    Bayesian approach (see Guindon, Guo, Welch, Theoretical Population Biology,
    2016; but also Koskela's PhD thesis
    \url{http://wrap.warwick.ac.uk/88065/1/WRAP_Theses_Koskela_2016.pdf}).  The
    spatial Lambda-Fleming-Viot model is mechanistic and enjoys many good
    statistical and biological properties (e.g., it does not suffer from the
    ``clumping'' issue that affects the Wright-Malecot model).
\end{point}

\reply{
    Certainly! We agree that the SLFV model is a promising way forward
    to be able to use coalescent simulations in continuous space.
    There's a lot of work left to explore how and when it's a good approximation
    to reality also,
    but we've added a bit about it to the ``Other methods'' section. \revref
}

\begin{point}{}
    Assuming that space is discrete in arguably the biggest weakness of the structured
    coalescent. It is quite often not biologically realistic and breaks down
    mathematically when considering larger numbers of smaller demes. The
    spatial Lambda-Fleming-Viot avoids this important shortcoming and thus
    probably deserves more attention.
\end{point}

\reply{
    We agree, and very much hope to move in that direction.
    We've added more discussion of this. \revref
}

